\documentclass[12pt, a4paper]{article}
\usepackage{charter}
\usepackage{setspace}
\onehalfspacing
\usepackage{geometry}
\geometry{hmargin=2cm,vmargin=2.5cm}
\pagestyle{empty}

\usepackage{amssymb}
\usepackage{enumerate}

\usepackage{tabularx}

%\usepackage{longtable}
%\usepackage{caption} % commenting next two lines shows the desired output
%\captionsetup[longtable]{labelsep=period, labelfont=bf}% added
%\usepackage{booktabs}
%\usepackage{array}
%\renewcommand{\arraystretch}{1.5}
% Alignement en bas
%\newcolumntype{R}[1]{>{\raggedleft\arraybackslash}m{#1}}
%\newcolumntype{C}[1]{>{\centering\arraybackslash}m{#1}}
%\newcolumntype{L}[1]{>{\raggedright\arraybackslash}m{#1}}


\begin{document}
	\section*{Questionnaire}
	Dans notre étude sur les dégâts de la chenille légionnaire d'automne (CLA) sur le rendement du maïs, nous aimerions avoir des réponses sur quelques questions que nous nous posons.
	\begin{enumerate}
		\item A quel stade du maïs, la CLA commence t-elle à attaquer les plantes ?

		\begin{tabularx}{0.7\textwidth}{*{4}{X}l}
			$\square\ V_2$	& $\square\ V_3$	& $\square\ V_4$	& $\square\ V_5$	& $\square$ Autre (préciser)
		\end{tabularx}

		Il faut noter qu'un stade $V_i$ signifie que le maïs a déroulé entièrement $i$ feuilles.

		\item A quels stades du maïs, les dégâts de la CLA sur les feuilles et sur les épis sont plus sévères ?
		\begin{enumerate}[-]
			\item Sur les feuilles :

			\begin{tabularx}{0.9\textwidth}{*{8}{X}l}
				$\square$ VT	& $\square$ R1	& $\square\ V_5$	& $\square\ V_6$	& $\square\ V_7$ & $\square\ V_8$	& $\square\ V_9$	& $\square\ V_{10}$	& $\square$ Autre
			\end{tabularx}

			\item Sur les épis :

			\begin{tabularx}{0.9\textwidth}{*{8}{X}l}
				$\square\ V_3$	& $\square\ V_4$	& $\square\ V_5$	& $\square\ V_6$	& $\square\ V_7$ & $\square\ V_8$	& $\square\ V_9$	& $\square\ V_{10}$	& $\square$ Autre
			\end{tabularx}
		\end{enumerate}

		\item A partir de quel stade du maïs, les dégâts de la CLA sur les feuilles commencent par diminuer ?
		\begin{enumerate}[$\square$]
			\item
		\end{enumerate}

		\item Comment qualifier les dégâts causés pas la CLA sur les feuilles par rapport à ceux causés sur épis ?
		\begin{enumerate}[$\square$]
			\item
		\end{enumerate}

		\item Quels sont les premiers signes de la présence de la CLA dans un champs de maïs ?
		\begin{enumerate}[$\square$]
			\item
		\end{enumerate}

		\item Quelles sont les mesures de prévention utilisées ?
		\begin{enumerate}[$\square$]
			\item
		\end{enumerate}

		\item Quelles sont les mesures de contrôle utilisées ?
		\begin{enumerate}[$\square$]
			\item
		\end{enumerate}

		\item Quelles les associations de cultures utilisées et leur efficacité relative ?
		\begin{enumerate}[$\square$]
			\item
		\end{enumerate}

		\item Pendant quelle saison l'invasion de la CLA sévit le plus ?
		\begin{enumerate}[$\square$]
			\item
		\end{enumerate}

		\item Quels sont les ennemis naturels de la CLA que vous connaissez ?
		\begin{enumerate}[$\square$]
			\item
		\end{enumerate}

		\item La CLA est-elle le seul ravageur du maïs ? si non, citez-en quelques uns et dire si ce sont les feuilles et/ou les épis que ceux-ci attaquent
		\begin{enumerate}[$\square$]
			\item
		\end{enumerate}
	\end{enumerate}

\noindent\textbf{NB :} Les stades peuvent être estimés en nombre de jours après la plantation, mais pas par le stade végétatif et reproductif.

	%\begin{!ongtable}{|L{0.3\textwidth}|L{0.2\textwidth}|L{0.2\textwidth}|L{0.3\textwidth}|}
	%	\toprule
	%
	%	\midrule
	%	\endhead
	%	\bottomrule
	%	\endfoot
	%\end{!ongtable}
\end{document}