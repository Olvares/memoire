%************************** COULEUR DES BLOCKS PERSONNALISES ***********************
\setbeamercolor{titre}{bg=red,fg=white}
\setbeamercolor{texte}{bg=red!10,fg=black}
\beamerboxesdeclarecolorscheme{ap}{cyan}{cyan!10} % pour sheme = couleur
\beamerboxesdeclarecolorscheme{exemple}{black}{white}
\beamerboxesdeclarecolorscheme{rem}{orange}{gray!30}

%\input{Theoreme_personnalise.tex}
%***********************************************************************************
\hypersetup{pdfpagemode=FullScreen} % Commande permettant de mettre en mode fullpage dans un lecteur PDF
\setcounter{secnumdepth}{0}
% **********************************************************************************
%********************* COMMANDES CREEES ***************************************
	\def\Hy{\mathcal{H}}
	\def\R{\mathbb{R}}
	\def\N{\mathbb{N}}
	\def\Z{\mathbb{Z}}
	\def\F{\mathcal{F}}
	\newcommand{\mrm}[1]{\ \mathrm{#1}\ }
	\newcommand{\enu}{\ \ \quad\clubsuit}
	%\newcommand{\F}{\mathcal{F}}
%******************************************************************************

\author{Olivier \textsc{Adjagba}}
\title{Modélisation}
%\subtitle{Initiation au \LaTeX}
%\setbeamercovered{transparent}
%\setbeamertemplate{navigation symbols}{}
%\logo{images/wsc.png}
\institute{ENSGMM}
%\date{}
%\subject{ok super}

%*************************** PAGE AVANT SECTION/SOUS-SECTION ************************
\AtBeginSection[]{
	\begin{frame}
		\frametitle{Sommaire}
		%\transwipe[duration=0.5]
		\tableofcontents[sectionstyle = show/shaded, subsectionstyle = hide]
	\end{frame}
}
\AtBeginSubsection{
	\begin{frame}
		\frametitle{Sommaire}
		%\transsplithorizontalin[duration=0.5]
		\tableofcontents[sectionstyle = show/hide, subsectionstyle = show/shaded/hide]
	\end{frame}
}
% ***********************************************************************************

\renewcommand{\arraystretch}{1.3}
% Alignement en bas
\newcolumntype{R}[1]{>{\raggedleft\arraybackslash}m{#1}}
\newcolumntype{C}[1]{>{\centering\arraybackslash}m{#1}}
\newcolumntype{L}[1]{>{\raggedright\arraybackslash}m{#1}}

% Renommer le caption table : \usepackage{caption}
\captionsetup[table]{name=Tableau}