%------------------------------------------------
%	PACKAGES AND OTHER DOCUMENT CONFIGURATIONS
%------------------------------------------------

\documentclass[
	12pt, % The default document font size, options: 10pt, 11pt, 12pt
	%oneside, % Two side (alternating margins) for binding by default, uncomment to switch to one side
	french, % ngerman for German
	onehalfspacing, % singlespacing, % Single line spacing, alternatives: onehalfspacing or doublespacing
	%draft, % Uncomment to enable draft mode (no pictures, no links, overfull hboxes indicated)
	%nolistspacing, % If the document is onehalfspacing or doublespacing, uncomment this to set spacing in lists to single
	liststotoc, % Uncomment to add the list of figures/tables/etc to the table of contents
	toctotoc, % Uncomment to add the main table of contents to the table of contents
	%parskip, % Uncomment to add space between paragraphs
	%nohyperref, % Uncomment to not load the hyperref package
	headsepline, % Uncomment to get a line under the header
	%chapterinoneline, % Uncomment to place the chapter title next to the number on one line
	consistentlayout, % Uncomment to change the layout of the declaration, abstract and acknowledgements pages to match the default layout
]{settings/MastersDoctoralThesis} % The class file specifying the document structure

%\usepackage[T1]{fontenc} % Requis
%%% Pour le document
%\usepackage{times} % Police serrée

%\usepackage{newcent} % Police aves des lettres un peu grandes

%\usepackage{palatino} % Police aves des lettres grandes
%\usepackage{bookman} % Police aves des lettres grandes un peu plus que palatino

\usepackage{chancery} % Police de manuscrit très élégant


%% Police additionnelle
%\usepackage{helvet} % Ne modifie que les sans sérif sf
%\usepackage{avant} % Idem que helvet mais un affichage différent
%\renewcommand{\familydefault}{\sfdefault} % Pour changer la police du document en sf

%\usepackage{courier} % Ne modifie que les tt
%\renewcommand{\familydefault}{\ttdefault} % % Pour changer la police du document en sf

%%% Pour les environnements mathématiques
%% ne disposant pas de sf et tt
%\usepackage{fourier}
%\usepackage{fouriernc} % Un peu différent de fourier
%\usepackage{mathptmx} % Pas de gras et certains symboles n'existent pas
%\usepackage{mathpazo} % Plus complete que mathptmx mais un peu moins élégant
%\usepackage[charter]{mathdesign} % Très élégant
%\usepackage[utopia]{mathdesign} % idem alternative de fourier
%% A ajouter au précédents
%\usepackage[scaled=0.875]{helvet} % Ajout de sf avec réduction d'échelle
%\usepackage{courier} % Ajout de tt

%% disposant pas de sf et tt
%\usepackage{txfonts} % Doit être chargé avant amsmath
%\usepackage{pxfonts} % idem
%\usepackage{kpfonts} % Pour palatino (sf)
%\usepackage{cmbright} % Bon pour présentation beamer
%\usepackage{lxfonts} % idem Mais moins bon à mon avis
\usepackage{beton, euler} % Cool aussi

%%% Font particulier
%\usepackage{frcursive} % Utiliser \begin{cursive} et \end{cursive} pour avoir une ecriture cursive
\usepackage[inlinechap]{settings/fncystyle}
\usepackage[utf8]{inputenc} % Required for inputting international characters
\usepackage[T1]{fontenc} % Output font encoding for international characters
\usepackage{verbatim}
%\usepackage{mathpazo} % Use the Palatino font by default

%\usepackage[backend=bibtex,style=authoryear,natbib=true]{biblatex} % Use the bibtex backend with the authoryear citation style (which resembles APA)

\usepackage[
	backend = biber,        % compilateur par défaut pour biblatex
	sorting = nyt,          % trier par nom, année, titre
	citestyle = authoryear-icomp, % style de citation auteur-année
	bibstyle = authoryear,  % style de bibliographie alphabétique
	firstinits = true, 		% pour afficher tous les prénoms et noms intermédiaires sous forme d’initiales
	maxcitenames = 2,
	maxbibnames = 50,
	%doi = false,
	%isbn = false,
	%url = false,
	%eprint = false
]{biblatex}
\addbibresource{BibGls/Biblio.bib}

%\addbibresource{example.bib} % The filename of the bibliography

\usepackage[autostyle=true]{csquotes} % Required to generate language-dependent quotes in the bibliography

%-------------------------------------
%	MARGIN SETTINGS
%-------------------------------------

\geometry{
	paper=a4paper, % Change to letterpaper for US letter
	inner=2.5cm, % Inner margin
	outer=3.8cm, % Outer margin
	bindingoffset=.5cm, % Binding offset
	top=1.5cm, % Top margin
	bottom=1.5cm, % Bottom margin
	%showframe, % Uncomment to show how the type block is set on the page
}

%----------------------------------------------------------------------------------------
%	THESIS INFORMATION
%----------------------------------------------------------------------------------------

\thesistitle{Modélisation des relations entre dégâts de \glsentrytext{cla}, \gls{spodo} et le rendement du maïs au Bénin} % Your thesis title, this is used in the title and abstract, print it elsewhere with \ttitle
\supervisor{Dr. } % Your supervisor's name, this is used in the title page, print it elsewhere with \supname
\examiner{} % Your examiner's name, this is not currently used anywhere in the template, print it elsewhere with \examname
\degree{Ingénieur de conception} % Your degree name, this is used in the title page and abstract, print it elsewhere with \degreename
\author{Olivier Mahumawon \textsc{Adjagba}} % Your name, this is used in the title page and abstract, print it elsewhere with \authorname
\addresses{} % Your address, this is not currently used anywhere in the template, print it elsewhere with \addressname

\subject{Génie Mathématiques et Modélisation} % Your subject area, this is not currently used anywhere in the template, print it elsewhere with \subjectname
\keywords{} % Keywords for your thesis, this is not currently used anywhere in the template, print it elsewhere with \keywordnames
\university{\href{http://https://unstim.bj}{Université Nationale des Sciences, Technologie, Ingénierie et Mathématiques}} % Your university's name and URL, this is used in the title page and abstract, print it elsewhere with \univname
\department{\href{http://department.university.com}{Ecole Nationale Supérieure de Génie Mathématiques et Modélisation}} % Your department's name and URL, this is used in the title page and abstract, print it elsewhere with \deptname
\group{\href{http://researchgroup.university.com}{Research Group Name}} % Your research group's name and URL, this is used in the title page, print it elsewhere with \groupname
\faculty{\href{http://faculty.university.com}{Faculty Name}} % Your faculty's name and URL, this is used in the title page and abstract, print it elsewhere with \facname

\AtBeginDocument{
	\hypersetup{pdftitle=\ttitle} % Set the PDF's title to your title
	\hypersetup{pdfauthor=\authorname} % Set the PDF's author to your name
	\hypersetup{pdfkeywords=\keywordnames} % Set the PDF's keywords to your keywords
}


%================ADDED===============
\usepackage{tabularx}
\usepackage{fourier-orns}
\usepackage{tikz}
\usetikzlibrary{arrows,calc,positioning,shapes.misc}
\usepackage{xspace}
%\usepackage{setspace}
% Pour les paragraphes
\parindent = 0pt
\parskip = 7pt
%\setstretch{1.1}
%\onehalfspacing
\setcounter{secnumdepth}{3}
\mtcsetdepth{minitoc}{1}
%\mtcsetdepth{minitoc}{0}
%\setcounter{minitocdepth}{1}

%Add this command where you want to print the Toc, Lof and Lot
\newcommand{\minitoclt}{
	\begingroup
	%\mtcsetdepth{minitoc}{1}
		\setcounter{tocdepth}{1}
		\newcommand{\mytocrule}{\hrule}
		\etocsettocstyle{{\bf\large Sommaire}
			\medskip%
			\mytocrule%
		}{%
			\smallskip%
			\mytocrule%
		}
		\localtableofcontents % \minitoc
		%\minilof
		%\minilot
	\endgroup
	\vskip 2\baselineskip
}

\def\dx{\mathrm{d}}
\newcommand{\dxdt}[2][t]{\frac{\dx#2}{\dx#1}}
\newcommand{\ddxdt}[2][t]{\displaystyle\dxdt[#1]{#2}}
\newcommand{\xp}[1]{\stackrel{.}{#1}}

\def\R{\mathbb{R}}
\def\N{\mathbb{N}}
\def\Z{\mathbb{Z}}

\usepackage{theorem}
%\theoremheaderfont{\bfseries}
%\theorembodyfont{\itshape}
\theoremstyle{break}
\newtheorem{Theo}{Théorème }[section]
\newtheorem{Prop}{Propriété }[section]
\newtheorem{Def}{Définition }[section]
\theorembodyfont{\rmfamily}
\newtheorem{Rem}{Remarque}[section]
\newtheorem{Ex}{Exemple}[section]
%=======================================


%%%%%%%%%%%%%%%%  inclure la source %%%%%%%%%%%%%%%%%%%%
\usepackage{listings}
\usepackage{inconsolata} % Swedish encoding in lstlisting

\newcommand*\styleC{\fontsize{9}{10pt}\usefont{T1}{pcr}{m}{n}\selectfont } % Change aussi la police avec {ptm}
\newcommand*\styleD{\fontsize{9}{10pt}\usefont{OT1}{pcr}{m}{n}\selectfont }% Change la police avec {pcr} et machine à écrire

%\makeatletter
%% on fixe le langage utilisé
%\lstset{language=matlab}
%\edef\Motscle{emph={\lst@keywords}}
%\expandafter\lstset\expandafter{%
%	\Motscle}
%\makeatother


\definecolor{Ggris}{rgb}{0.45,0.48,0.45}

\renewcommand{\lstlistingname}{Algorithme}
\lstset{frameround=fttt}
%\lstset{backgroundcolor=\color{gray!20}}
%% Pour les codes
%\lstset{emphstyle=\bfseries\ttfamily\color{blue!70!black}, % les mots réservés de matlab en bleu
	%	basicstyle=\fontsize{10}{12pt}\styleD,
	%	keywordstyle=\ttfamily,
	%	commentstyle=\color{blue}\styleC, % commentaire en vert
	%	numberstyle=\tiny\color{red}, % Couleur de la numérotation des lignes de codes
	%	numbers=none,
	%	numbersep=10pt,
	%	lineskip=0.7pt,
	%	frame = trBL,
	%	title= {Code R},
	%	showstringspaces=false}
%%  % inclure le fichier source
%\newcommand{\R}[1]{%
	%	\lstinputlisting[texcl=true]{#1}
	%}

\lstdefinestyle{pos}{float=!ht}

\lstset{
%	backgroundcolor=\color{gray!10},
%	emphstyle=\bfseries\ttfamily\color{blue!70!black}, % les mots réservés de matlab en bleu et gras
%	basicstyle=\fontsize{10}{12pt}\styleD,
%	keywordstyle=\ttfamily,
%	commentstyle=\color{green!40!black}\styleC, % commentaire en vert
%	numberstyle=\tiny\color{red}, % Couleur de la numérotation des lignes de codes
%	numbers=left,
	%numberfirstline,
%	numbersep=10pt,
%	lineskip=0.7pt,
	breaklines,
	frame = tb,
	%float=ht, %inutile
%	title= {Code Matlab},
%	showstringspaces=false,
	escapechar=\%,
	mathescape,
	tabsize=4,
	inputencoding = utf8,  % Input encoding
	extendedchars = true,  % Extended ASCII
	literate      =        % Support additional characters
	{á}{{\'a}}1  {é}{{\'e}}1  {í}{{\'i}}1 {ó}{{\'o}}1  {ú}{{\'u}}1
	{Á}{{\'A}}1  {É}{{\'E}}1  {Í}{{\'I}}1 {Ó}{{\'O}}1  {Ú}{{\'U}}1
	{à}{{\`a}}1  {è}{{\`e}}1  {ì}{{\`i}}1 {ò}{{\`o}}1  {ù}{{\`u}}1
	{À}{{\`A}}1  {È}{{\'E}}1  {Ì}{{\`I}}1 {Ò}{{\`O}}1  {Ù}{{\`U}}1
	{ä}{{\"a}}1  {ë}{{\"e}}1  {ï}{{\"i}}1 {ö}{{\"o}}1  {ü}{{\"u}}1
	{Ä}{{\"A}}1  {Ë}{{\"E}}1  {Ï}{{\"I}}1 {Ö}{{\"O}}1  {Ü}{{\"U}}1
	{â}{{\^a}}1  {ê}{{\^e}}1  {î}{{\^i}}1 {ô}{{\^o}}1  {û}{{\^u}}1
	{Â}{{\^A}}1  {Ê}{{\^E}}1  {Î}{{\^I}}1 {Ô}{{\^O}}1  {Û}{{\^U}}1
	{œ}{{\oe}}1  {Œ}{{\OE}}1  {æ}{{\ae}}1 {Æ}{{\AE}}1  {ß}{{\ss}}1
	{ç}{{\c c}}1 {Ç}{{\c C}}1 {ø}{{\o}}1  {å}{{\r a}}1 {Å}{{\r A}}1
	{ñ}{{\~n}}1  {Ñ}{{\~N}}1  {¿}{{?`}}1  {¡}{{!`}}1   {'}{{'}}1
	{°}{{\textdegree}}1 {º}{{\textordmasculine}}1 {ª}{{\textordfeminine}}1
	% ¿ and ¡ are not correctly displayed if inconsolata font is used
	% together with the lstlisting environment. Consider typing code in
	% external files and using \lstinputlisting to display them instead.
}
% inclure le fichier source
%\newcommand{\Matlab}[1]{%
%	\lstinputlisting[texcl=true]{#1}
%} % A Revoir
