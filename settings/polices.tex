%\usepackage[T1]{fontenc} % Requis
%%% Pour le document
%\usepackage{times} % Police serrée

%\usepackage{newcent} % Police aves des lettres un peu grandes

%\usepackage{palatino} % Police aves des lettres grandes
%\usepackage{bookman} % Police aves des lettres grandes un peu plus que palatino

\usepackage{chancery} % Police de manuscrit très élégant


%% Police additionnelle
%\usepackage{helvet} % Ne modifie que les sans sérif sf
%\usepackage{avant} % Idem que helvet mais un affichage différent
%\renewcommand{\familydefault}{\sfdefault} % Pour changer la police du document en sf

%\usepackage{courier} % Ne modifie que les tt
%\renewcommand{\familydefault}{\ttdefault} % % Pour changer la police du document en sf

%%% Pour les environnements mathématiques
%% ne disposant pas de sf et tt
%\usepackage{fourier}
%\usepackage{fouriernc} % Un peu différent de fourier
%\usepackage{mathptmx} % Pas de gras et certains symboles n'existent pas
%\usepackage{mathpazo} % Plus complete que mathptmx mais un peu moins élégant
%\usepackage[charter]{mathdesign} % Très élégant
%\usepackage[utopia]{mathdesign} % idem alternative de fourier
%% A ajouter au précédents
%\usepackage[scaled=0.875]{helvet} % Ajout de sf avec réduction d'échelle
%\usepackage{courier} % Ajout de tt

%% disposant pas de sf et tt
%\usepackage{txfonts} % Doit être chargé avant amsmath
%\usepackage{pxfonts} % idem
%\usepackage{kpfonts} % Pour palatino (sf)
%\usepackage{cmbright} % Bon pour présentation beamer
%\usepackage{lxfonts} % idem Mais moins bon à mon avis
\usepackage{beton, euler} % Cool aussi

%%% Font particulier
%\usepackage{frcursive} % Utiliser \begin{cursive} et \end{cursive} pour avoir une ecriture cursive