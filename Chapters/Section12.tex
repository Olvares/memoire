\section{La \glsentrylong{cla}}
%\section{Introduction}
La \glsentrylong{cla} (\glsentrytext{spodo}) est une espèce de lépidoptères (papillons) de la famille des \emph{Noctuidae}. C’est un ravageur des cultures céréalières extrêmement dévastateur. Son intrusion dans un champs de ses cultures préférées provoque des dégâts énormes en un temps record. %Cet insecte est classifié comme indiqué dans le tableau~\ref{classi:spodo}.
Dans cette section, nous parlerons

%\begin{table}
%	\centering
%	\caption{Classification du \glsentrytext{spodo}}
%	\label{classi:spodo}
%	\begin{tabular}{>{\bf}l>{\it}l}
	%		\toprule
	%		Règne					& Animalia			\\
	%		Embranchement			& Arthropoda		\\
	%		Sous-embr.				& Hexapoda			\\
	%		Classe					& Insecta			\\
	%		Sous-classe				& Pterygota			\\
	%		Infra-classe			& Neoptera			\\
	%		Ordre					& Lepidoptera		\\
	%		Super-famille			& Noctuoidea		\\
	%		Famille					& Noctuidae			\\
	%		Sous-famille			& Hadeninae			\\
	%		Tribu					& Caradrinini		\\
	%		Genre					& Spodoptera		\\
	%		\midrule
	%		\multicolumn{2}{c}{\bf Espèce}				\\
	%		\multicolumn{2}{c}{\glsentrytext{spodo}}	\\
	%		\multicolumn{2}{c}{J. E. Smith, 1797}		\\
	%		\bottomrule
	%	\end{tabular}
%\end{table}


\subsection{Origine et historique du \glsentrytext{spodo} (\glsentrylong{cla})}
%La \gls{cla} est originaire des régions tropicales d’Amérique, des Etats-Unis à l’Argentine et aux Caraïbes. Comme elle ne dispose d’aucun mécanisme de diapause, elle ne peut passer l’hiver que dans des climats doux du sud de la Floride et du Texas ; chaque année, elle envahit une grande partie du territoire continental des Etas-Unis et du sud du Canada \autocite{sparks1979}. En Afrique, le ravageur a été découvert pour la première fois %au Nord Ouest du Nigeria
%en 2016 \autocite{goergen2016first}.

Les apparitions de la \gls{cla} se sont produites aux États-Unis à des intervalles très irréguliers. L'espèce a été enregistrée comme un ravageur néfaste en Géorgie dès 1797 \autocite{smith1797natural}. La prochaine épidémie a été signalée par \cite{Glover1856} qui a déclaré que les «marching-worm» causaient des dommages dans l'ouest de la Floride suffisamment importants pour justifier l'institution de la \og méthode de fossé\fg\ pour détruire les vers.

\cite{luginbill1928fall} a noté de nombreux rapports sur les épidémies de la \gls{cla} aux États-Unis de 1856 à 1928. Des dommages graves au maïs ont été signalés au le Missouri et en Illinois en 1870. Des dommages moins importants se sont alors produits peu fréquemment jusqu'en 1899 lorsqu'une épidémie a été signalée des Carolines de l'ouest jusqu'au Kansas et au Missouri. \cite{walton1936fall} ont rapporté qu'une épidémie particulièrement grave s'est produite à l'été 1912 lorsque le ravageur a balayé presque l'ensemble des Etats-Unis à l'est des montagnes rocheuses, a complètement détruit le maïs et le mil dans le sud des Etats-Unis, des cultures de coton% et de camion gravement blessées,
et a détruit des herbes sur les pelouses dans les villes comme par magie. \cite{luginbill1928fall} a signalé d'autres épidémies en 1912, 1915, 1918 et 1920. Depuis 1928, plusieurs épidémies de la \gls{cla} se sont produites. Ces dernières ne sont pas bien documentées, mais les entomologistes se souviennent des années où la \gls{cla} est devenue un ravageur général, prédominant et nuisible  ayant dévasté des cultures dans certaines régions du sud. Les entomologistes se souviendront surtout de 1975, 1976, 1977 comme des années de fortes infestations de \gls{cla} dans tout le sud et le long de la côte atlantique \autocite{sparks1979}. % Bien qu'il n'y ait pas de méthode exacte pour quantifier les pertes ou les valeurs des cultures pour la comparaison, l'année 1977 rivale certainement toute autre année. Les pertes estimées en dollars attribuées à la \gls{cla} sur toutes les cultures dans le sud-est des états-unis en 1975 et 1976 étaient de 61.2 et 31.9 millions de dollars, respectivement; Les pertes de 1977 en Géorgie seulement ont été estimées à 137.5 millions de dollars.


\subsection{Biologie du \glsentrytext{spodo}}\label{revue:cla:bio}
Le cycle biologique de la \gls{cla} comprend l'œuf, 6 étapes de développement de la chenille (stades larvaires), la pupe et la noctuelle (adulte) \autocite{fao201812}.

\begin{figure}
	\centering
	\begin{subfigure}[b]{0.3\textwidth}
		%\centering
		\includegraphics[width = \textwidth, height = 3.5cm]{egg}
		\subcaption{Masse d’œufs}
		\label{cla:egg}
	\end{subfigure}
	%~
	\begin{subfigure}[b]{0.3\textwidth}
		%\centering
		\includegraphics[width = \textwidth, height = 3.5cm]{jlarv}
		\subcaption{Jeunes larves}
		\label{cla:jlarv}
	\end{subfigure}
	\begin{subfigure}[b]{0.3\textwidth}
		%\centering
		\includegraphics[width = \textwidth, height = 3.5cm]{fivstlarv}
		\subcaption{Larves du stade 1 à 5}
		\label{cla:fivstlarv}
	\end{subfigure}

	\begin{subfigure}[b]{0.3\textwidth}
		%\centering
		\includegraphics[width = \textwidth, height = 3.5cm]{adlarv}
		\subcaption{Larves au stade 6}
		\label{cla:adlarv}
	\end{subfigure}
	%~
	\begin{subfigure}[b]{0.3\textwidth}
		%\centering
		\includegraphics[width = \textwidth, height = 3.5cm]{pupe}
		\subcaption{Pupe}
		\label{cla:pupe}
	\end{subfigure}
	\begin{subfigure}[b]{0.3\textwidth}
		%\centering
		\includegraphics[width = \textwidth, height = 3.5cm]{papi}
		\subcaption{Noctuelle}
		\label{cla:papi}
	\end{subfigure}

	\caption[Stades du cycle de vie du \glsentrytext{spodo}]{Stades du cycle de vie du \glsentrytext{spodo}\footcite{fao201812}}
	\label{cla:lcstage}
\end{figure}

Dans des conditions favorables, la \gls{cla} femelle pond $1500$ à $2000$ œufs pendant son cycle de vie. Les œufs éclosent en larves néonates 4 jours après avoir été pondus \autocite{SimmonsLynch1998, prasanna2018fall}. Le stade larvaire se compose de six étapes, suivies d'une étape pré-pupe, au cours de laquelle la larve tombe et s'enfouit dans le sol à une profondeur de $7,62$ à $10,16$\,cm pendant $2$ à $4$ jours avant la transformation en pupe \autocite{hardke2015}. Cette étape de transformation en pupe peut durer de $7$ à $14$ jours, selon la température du sol. Les insectes émergents du stade de pupe se déplacent vers la surface du sol, deviennent adultes et infestent des plantes en croissance. Le cycle de vie entier d'une \gls{cla} peut prendre jusqu'à 4 semaines, tandis qu'une génération composée de larves émergeant de masses d'œufs d'âge similaire peut durer de 80 à 90 jours \autocite{Abrahams2017}. Le cycle de vie, la reproduction et la distribution de la \gls{cla} dépendent des conditions de température, de la saison des cultures, de la présence d'espèces végétales hôtes et de la fécondité. La propagation rapide du ravageur en \ass est probablement liée à la capacité de dispersion notable du papillon, aux conditions éoliennes prévalantes et à la disponibilité d'espèces hôtes variées \autocite{chapaw2021}.% Les papions de la \gls{cla} sont nocturnes et sont capables de voler sur 100 km en une nuit \autocite{Zhou2020}. Jeger et coll. (2018) a signalé un vol à distance par d'autres ravageurs nocturnes liés à la \gls{cla}, y compris S. exigua, d'Afrique du Nord, atteignant le Royaume-Uni et l'Espagne. La \gls{cla} préfère les hôtes d'herbe, tels que le maïs, le sor-ghum et l'herbe bermuda.
Les larves plus âgées présentent une alimentation nocturne et vorace qui atteint son pic au cours des deux dernières stades larvaires \autocite{HARRISON2019318}.

Il existe deux biotypes connus de la \gls{cla}, à savoir : les biotypes de maïs (C) et de riz (R) \autocite{Abrahams2017}. Les deux biotypes ont été distingués par l'utilisation de marqueurs \gls{snp} \autocite{sibanda2018}. Morphologiquement, les biotypes sont identiques \autocite{nagoshi2018}, mais leurs étapes de développement diffèrent en taux de croissance, en poids nymphal (poids de la pupe) et en période de ponte \autocite{Pashely1988}. De plus, on croyait que les deux biotypes étaient sexuellement incompatibles \autocite{Pashely1988}. Cependant, l'accouplement réussi des femelles de R-biotype avec les mâles de C-biotype a été signalé, bien que les croix réciproques aient échoué \autocite{Pashely1988}. Les compatibilités de rotation entre les deux biotypes peuvent entraîner des variantes nouvelles et agressives. Les deux biotypes sont génétiquement et comportementalement différents, et donc ils ont besoin de différentes méthodes de contrôle. \cite{Kuate2019} a signalé une réponse différente des biotypes de la \gls{cla} aux produits chimiques de protection des cultures, avec un taux de mortalité différent.

%La figure \ref{cla:lcstage} montre les stades successifs du cycle de vie dela \gls{cla}.

\subsection{Plantes hôtes du \glsentrytext{spodo}}
La \gls{cla} est un ravageur polyphage ; c’est-à-dire qu’il attaque un grand nombre de plants mais a une préférence pour les graminées.

Malgré la large gamme d'hôtes qu'a la \gls{cla}, ses deux biotypes préfèrent principalement le maïs et le riz \autocite{prasanna2018fall,Casmuz2017} a signalé que la \gls{cla} a causé des dommages variables sur les familles suivantes de plantes dans l'ordre décroissant de proportions : \bact{Poaceae} ($35,5\%$), \bact{Fabaceae} ($11,3\%$), \bact{Solanaceae} et \bact{Asteraceae} ($4,3\%$), \bact{Rosaceae} et \bact{Chenopodiaceae} ($3,7\%$) et \bact{Brassicacae} et \bact{Cyperaceae} ($3,2\%$). Par conséquent, la famille \bact{Poaceae}, qui contient les principales cultures céréalières, y compris le maïs, sert de principal hôte de la \gls{cla}. Au cours de l'étape larvaires, la \gls{cla} acquiert \og l'habitude de la légionnaire\fg\ et se propage souvent en grand nombre, défoliant agressivement ses plantes hôtes \autocite{Abrahams2017}. Les hôtes primaires et alternatifs de la \gls{cla}, y compris les céréales, comme le blé et les espèces d'herbe qui poussent pendant les périodes hors saison, étendent ses chances de survie, servant de ponts verts entre les espèces végétales et la période de culture \autocite{prasanna2018fall}. %Systèmes monocropping pratiqués par de nombreux agriculteurs en \ass exacerber le problème de la \gls{cla} (Midega et coll. 2018).

%Figure 1. Carte de l'Afrique montrant la propagation de la \gls{cla}. Actuellement, 45 pays (couleur verte) ont confirmé la présence de la \gls{cla}, la Guinée équatoriale a soupçonné la présence et les six pays en jaune, dont la Mauritanie, le Maroc, la Tunisie, l'Algérie, la Libye et le Lesotho n'ont pas signalé de ravageurs

\subsection{Distribution géographique du \glsentrytext{spodo} en Afrique}
La première apparition de la \gls{cla} en Afrique a été signalée précisément en Janvier 2016 sur des plants de maïs dans la zone tropicale au Nord-Ouest du Nigeria et dans des domaines de maïs à \gls{iita} à Ibadan et Ikenne \autocite{goergen2016first}. Avant la fin de cette année (2016), d'autres pays tels que le São Tomé et Príncipe, le Bénin, le Togo ont également signalé la présence du ravageur. %(voir figure \ref{cla:distNov16}).
La \gls{cla} peut migrer sur des centaines de kilomètres par nuit et se reproduit tous les 1–2 mois ; ce qui fait que ce ravageur se propage aussi rapidement en Afrique depuis son apparition \autocite{Stokstad2017}. %C’est ainsi que ; déjà en Avril 2017 ; soit un environ 16 mois après son apparition, le ravageur s’est retrouvé  dans plus d'une vingtaine de pays d’Afrique sub-saharienne (voir figure \ref{cla:distApr17}). Comme l'indique la figure \ref{cla:distFeb18}, en Février 2018, la quasi-totalité des pays d’Afrique sub-saharienne ont été touchés.
Ainsi, en avril 2017, au moins 16 pays africains avaient confirmé la présence de la \gls{cla}, et plus tard en octobre, plus de 30 pays africains avaient confirmé la présence de la \gls{cla} \autocite{Abrahams2017,fao201812}. En décembre 2018, 41 pays africains sur 54 avaient confirmé la présence de la \gls{cla}, alors que trois pays, à savoir le Gabon, la Guinée équatoriale, et la République du Congo soupçonnaient sa présence, en attendant la confirmation \autocite{fao201806}.

Lesotho est le seul pays d'Afrique australe sans présence \gls{cla} \autocite{Rwomushana2018} Dans la conception de stratégies intégrées de gestion de la \gls{cla} pour l'\ass, une enquête détaillée sur les différents facteurs qui peuvent contribuer à entraver l'infestation et la distribution de la \gls{cla} au Lesotho est cruciale.%Ces informations peuvent également être utiles pour l'ue, qui a lancé une étude des zones sans \gls{cla} en afrique pour une production possible d'un marché de niche spécifique pour les produits agricoles (Jeger et coll. 2018).
Àu milieu de 2019, quatre pays de plus ont confirmé la présence de la \gls{cla}, apportant le nombre total des pays touchés en Afrique à 45 \autocite{Rwomushana2018}. Outre les pays d'Afrique continentale, la présence de la \gls{cla} est également confirmée dans les îles associées à l'Afrique, comme Madagascar \autocite{chapaw2021}.
% Au milieu de 2018, la \gls{cla} s'est répandue au-delà de l'Afrique vers l'Asie, où elle a été détectée pour la première fois dans la région de Chikkaballapur de l'État de Karnataka (CABI 2018). Il a depuis été détecté dans plus de 10 états en inde. Les derniers rapports ont documenté la présence confirmée de la \gls{cla} dans huit autres pays d'Asie, viz., Chine, Indonésie, Malaisie, Sri Lanka, Bangladesh, Myanmar, Vietnam et Thaïlande (Balla et coll. 2019).

La carte de la figure \ref{cla:dist} montre la distribution géographique de la \gls{cla} en mai 2021 dans le monde. On peut y voir qu'il n'a fallu que 2\,ans pour le ravageur pour envahir la quasi-totalité des pays de l'\ass.

\begin{figure}
	\centering
	\includegraphics[width = \textwidth]{SpodoDistMay21}
	\caption[Distribution géographique de la \glsentryshort{cla} de 2016 à 2021]{Distribution géographique de la \gls{cla} depuis son apparition en Afrique en 2016 jusqu'en Mai 2021 dans le monde\footcite{fao2021}}
	\label{cla:dist}
\end{figure}

%\begin{figure}
%	\centering
%	\begin{subfigure}{0.45\textwidth}
	%		%\centering
	%		\includegraphics[width = 0.9\textwidth]{SpodoDistJan16}
	%		\subcaption{Janvier 2016}
	%		\label{cla:dist16}
	%	\end{subfigure}
%	~
%	\begin{subfigure}{0.45\textwidth}
	%		%\centering
	%		\includegraphics[width = 0.9\textwidth]{SpodoDistNov16}
	%		\subcaption{Novembre 2016}
	%		\label{cla:distNov16}
	%	\end{subfigure}
%
%	\begin{subfigure}{0.45\textwidth}
	%		%\centering
	%		\includegraphics[width = 0.9\textwidth]{SpodoDistFeb17}
	%		\subcaption{Février 2017}
	%		\label{cla:distFeb17}
	%	\end{subfigure}
%	~
%	\begin{subfigure}{0.45\textwidth}
	%		%\centering
	%		\includegraphics[width = \textwidth]{SpodoDistApr17}
	%		\subcaption{Avril 2017}
	%		\label{cla:distApr17}
	%	\end{subfigure}
%
%	%\begin{subfigure}{0.4\textwidth}
%	%	%\centering
%	%	\includegraphics[width = \textwidth]{Spodo_dist2017_2}
%	%	\caption{2017-2}
%	%	\label{cla:dist17-2}
%	%\end{subfigure}
%	%~
%	%\begin{subfigure}{0.9\textwidth}
%	%	%\centering
%	%	\includegraphics[width = \textwidth]{SpodoDistFeb18}
%	%	\subcaption{Février 2018}
%	%	\label{cla:dist18}
%	%\end{subfigure}
%	%\caption{Distribution de la \gls{cla} de 2016 à 2018 en Afrique}
%	%\label{cla:dist}
%\end{figure}

%\begin{figure}
%	\ContinuedFloat
%	\centering
%	\begin{subfigure}{\textwidth}
	%		%\centering
	%		\includegraphics[width = 0.5\textwidth]{SpodoDistFeb18}
	%		\subcaption{Février 2018}
	%		\label{cla:distFeb18}
	%	\end{subfigure}
%	\caption{Distribution de la \gls{cla} de 2016 à 2018 en Afrique}
%	\label{cla:dist}
%\end{figure}


\subsection{Prévention et contrôle du \glsentrytext{spodo}}
%Depuis l'apparition de la \gls{cla}, a réaction immédiate a été l'utilisation des produits chimiques tels que les insecticides pour lutter contre ce ravageur. Or, l'utilisation de ces produits ne garantit pas la sécurité alimentaire et environnementale \autocite{thierfelder2018low}

Plusieurs stratégies de contrôle ont été recommandées, qui peuvent être testées et adoptées en \ass. Ces stratégies incluent les pratiques culturelles, le contrôle chimique \autocite{Abrahams2017}, les agents de biocontrôle, résistance des plantes d'hôte et gestion intégrée de la \gls{cla} \autocite{prasanna2018fall}. Les pratiques culturelles recommandées incluent la plantation en temps opportun en suivant les principales chutes de pluie, la rotation des cultures, de préférence avec des espèces non gazeuses, telles que le soja, la combustion des résidus de culture et la gestion du paysage en enlevant les hôtes majeurs et alternatifs autour des champs de maïs \autocite{Abrahams2017}. Les pratiques culturelles sont le point de départ pour minimiser les populations de ravageurs et peuvent être considérées comme des mesures préventives. Actuellement en \ass, l'application de pesticides chimiques serait la stratégie de contrôle la plus utilisée pour la \gls{cla} \autocite{Stokstad2017}. Certains agents de biocontrôle sont connus pour être des ennemies naturelles de la \gls{cla} en Amérique. \cite{Abrahams2017} a signalé un taux de réussite de 70\% dans le contrôle de la \gls{cla} en utilisant des guêpes parasitaires au Brésil.

Le déploiement réussi des agents de biocontrôle pour une utilisation en \ass nécessite des recherches ciblées sur les interactions entre les biotypes de la \gls{cla} et leurs ennemis naturels. Les cultures transgéniques offrent une option de contrôle de la \gls{cla} et sont la principale forme de contrôle utilisée contre le ravageur en Amérique \autocite{hruska2019}, mais ne sont utilisées que dans quelques pays d'Afrique car les systèmes agricoles sont souvent très différents de ceux en Afrique. Peu de régions d'Amérique ont les petites tailles de ferme et de terrain qui prédominent en Afrique. Les rendements sont beaucoup plus élevés en Amérique, en moyenne plus de huit tonnes par hectare pour le maïs, par rapport à environ deux tonnes par hectare en Afrique \autocite{day2017fall}. La résistance des plantes hôtes basée sur la capacité intrinsèque de la plante à résister ou à tolérer l'herbivorie \gls{cla}, est un approche économique et écologique utile aux petits exploitants et aux agriculteurs commerciaux. Elle repose sur les différents mécanismes de défense des plantes contre les insectes nuisibles. Ces systèmes de défense sont importants à prendre en compte lors de la sélection des parents pour l'élevage de résistance des plantes hôtes \autocite{chapaw2021}.

La gestion intégrée de la \gls{cla} comprend deux ou plusieurs stratégies de gestion qui peuvent fournir des effets synergiques sur la lutte antiparasitaire avant que des dommages importants ne soient infligés \autocite{fao201812}. Les départements de la recherche internationale, tels que le \glsentryfull{cimmyt}, \glsentryfull{iita}, \glsentryfull{cabi},\glsentryfull{icipe}, \glsentryfull{fao}, les instituts de recherches nationaux ainsi que le secteur privé travaillent ensemble activement pour élaborer des stratégies de contrôle rentable de la \gls{cla}. Le \gls{cimmyt} et \gls{iita} visent à introduire de nouveaux gènes de résistance \gls{cla} dans des variétés sensibles pour développer des cultures en \ass \autocite{chapaw2021}.
