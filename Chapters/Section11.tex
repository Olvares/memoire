\section[Biologie des plantes, insectes et de leurs interactions]{Biologie des plantes, insectes et de leurs interactions \footcite{lebon:tel-01747618}}

Les relations plantes-insectes ont toujours été à double tranchant. D’un côté de nombreux végétaux ne pourraient survivre sans la présence de leurs insectes pollinisateurs. De l’autre, une part non négligeable des récoltes est directement consommée par les insectes que cela soit au champs ou lors de leur stockage. Depuis toujours, les hommes se sont donc tout naturellement penchées sur ces relations afin de les comprendre mais aussi de les modifier, surtout depuis quelques années. En effet, avec pour objectif de nourrir plus de neuf milliards d’individus sur une planète qui de toute évidence a des limites de production, il est nécessaire de réduire au maximum les pertes subies afin d’améliorer les rendements finaux. Pour cela, l’homme a mis en place des méthodes de lutte diversifiée qui modifie ces interactions plantes-insectes : lutte chimique, mécanique, biologique ou encore intégrée. Afin d’optimiser ces méthodes et prédire leur impacts sur les interactions plantes-insectes, il est nécessaire de cerner au plus près ces relations plantes-insecte.

\subsection{L’objet Plante}
Les végétaux sont des organismes photoautotrophes, c’est à dire les producteurs primaires des chaines alimentaires (à l’exception des plantes parasites et des plantes carnivores). Ils sont capables de capturer l’énergie solaire sous forme de photons et la transformer en carbohydrates, des composés de matière organique. Ce phénomène est appelé photosynthèse. La matière organique ainsi créée peut être allouée à différents compartiments de la plante et être utilisée dans le cadre de la croissance (principalement par la respiration), dans la reproduction, ou être stockée. Cependant, elle peut également être détournée par des consommateurs secondaires tels que les insectes phytophages. Les plantes sont un groupe très diversifié, présentant de multiples formes et taille. Néanmoins, elles partagent toutes la même organisation de base dont les parties principales sont les feuilles o`u se déroule la photosynthèse, les tiges ou branches qui servent de support et les racines qui ancrent la plante au sol et permettent l’acquisition d’eau et de nutriments. Lors de la phase reproductive, de nouveaux organes apparaissent : les fleurs, les fruits mais aussi les organes de réserves. Cette organisation ainsi que les principaux mécanismes ayant lieu dans une plante sont représentés dans la figure 2.1. Il s’agit d’un schéma simplifié d’une plante o`u sont représentées les zone de photosynthèse (zone de production) et de respiration, les zones de stockage et de croissance, et les flux de matière sous la forme de sève brute et élaborée. Chacun de ces processus sera abordé dans cette section.

Les informations présentées dans cette section proviennent principalement des ouvrages de \autocite{Heller1998, Heller2000}.

\subsection{L’objet Insecte}
La classe des insectes, découpée en de très nombreux ordres, représente près de la moitié des espèces vivantes recensées au monde et près des trois quarts de celle du monde animal \autocite{Sauvion2013}. Si un million d’espèces a déjà été décrites, il en resterait près de 30 millions encore inconnus, présents dans toutes les zones du globe et sous tous les climats. C’est l’organisation du corps qui détermine l’appartenance d’un animal à la classe des insectes. Il doit comporter : une tête, porteuse des organes des sens (yeux et antennes classiquement) et de la nutrition (pièces buccales), d’un thorax o`u sont attachés les trois paires de pattes ainsi que les ailes et un abdomen qui contient les organes internes. Ce corps étant entièrement recouvert de chitine (ce qui constitue leur exosquelette), la respiration s’effectue par leur peau au travers de trachées ouvertes sur l’extérieur appelées des stigmates et disposées le long de l’abdomen. Si certains aspects de cette organisation tel que le nombre d’ailes peut être légèrement modifiés (chez les diptères par exemple, une paire d’ailes s’est atrophiée afin de créer ce qu’on appelle les balanciers), la présence des six pattes est obligatoire. C’est le critère essentiel de la détermination des insectes. Par exemple, les acariens qui sont souvent comparés aux insectes ont huit paires de pattes et sont donc des arachnides.

\subsection{Interactions entre les plantes et les insectes}
Quand les plantes et les insectes cohabitent, deux types d'interactions peuvent se produire entre eux. IL s'agit soit du l'antagonisme, soit du mutualisme. Nous présentons brièvement ces deux types d'interactions dans cette sous-section.

	\subsubsection{Antagonisme plantes-insectes}
	Les plantes et les insectes étant tous deux des éléments particulièrement complexes et affichant une incroyable diversité, les relations qu’ils ont pu mettre en place durant leur co-évolutions sont tout aussi riches. Il est néanmoins possible de les séparer en deux grandes catégories qui sont les relations antagonistes o`u plantes et insectes s’opposent, tentant chacun de survivre, et les relations mutualistes o`u des interactions bénéfiques pour ls deux protagonistes apparaissent. Nous commencerons ici par les relations antagonistes car ce sont elles qui ont motivés les hommes dans la recherche de moyens de lutte tels que la lutte biologique, la lutte chimique ou encore la sélection variétale en vue de trouver des plantes plus résistantes. Au vue de la diversité des deux groupes, les relations antagonistes entre plante et insecte peuvent prendre de nombreuses formes. L’herbivorie peut ainsi diminuer la surface foliaire, mais également les stocks de nutriments, la capacité photosynthétique \autocite{Zangerl1088}, la capacité reproductrice \autocite{Quesada1995} ou encore la croissance végétative \autocite{Meyer1998}. Ces différents types d’attaques conduisent à des pertes quantitatives (réduction de la production de biomasse) et/ou à des pertes qualitatives (diminution du contenu nutritionnel, de la qualité commerciale ou de la caractéristique de stockage,\dots \autocite{Oerke2006crop}). Celles-ci peuvent être exprimées en termes de valeurs absolues (kg/ha, ou valeur financière/ha) ou de valeurs relatives (\%) \autocite{Oerke2006crop}.

	\subsubsection{Mutualisme plantes-insectes}
	Cependant, les relations plantes-insectes ne se résument pas à des interactions de type antagonistes. Nombreuses sont les plantes qui ne pourraient se reproduire, se disperser voire survivre sans la présence d’insectes. Il s’agit dont de relations mutualistes. Le mutualisme est défini comme un interaction bénéfique entre deux individus (ou deux populations) d’un point de vue reproduction et/ou survie. Cette relation peut être obligatoire (les deux protagonistes ne peuvent pas survivre l’un sans l’autre) ou facultative \autocite{Holland2008}. Charles Darwin a probablement été le premier à mettre en avant le mutualisme et plus particulièrement la pollinisation, le désignant comme une exception à sa théorie de l’évolution. Mais le mutualisme peut prendre d’autres formes telles que le mutualisme de protection, de dispersion ou même de nutrition. Cependant les limites du mutualisme restent floues et ce qui apparait comme d’une mutualisme à une échelle large peut être en fait, tout autre lorsque l’on change d’échelle. Dans cette section, nous passerons en revue ces différents types de mutualisme que sont le mutualisme de pollinisation, de protection, de dispersion, de nutrition mais également un dernier type de mutualisme fondé sur une base d’antagonisme et une réaction de défense de plante très particulière appelée tolérance.