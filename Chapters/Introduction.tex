\chapter{Introduction}
%\addstarredchapter{Introduction}
\minitoclt
%\setcounter{minitocdepth}{2}
%\minitoc

%\lhead{Introduction}
\section{Introduction}
\subsection{Contexte et justification}
%Mêlant à la fois la complexité des plantes et la variété du monde des insectes, les interactions entre plantes et insectes est un domaine extrêmement riche. Parmi ces interactions, la pollinisation est indubitablement celle à laquelle on pense en premier. Et pour cause, les insectes permettent l’existence et la reproduction de nombreuses plantes. La pollinisation est considérée comme l’un des services écosystémiques les plus répandus au monde. Selon le programme de recherche européen, Alarm \footnote{\url{.http ://www.alarmproject.net}}, mené de 2004 à 2009 pour évaluer les risques encourus par la biodiversité terrestre et aquatique en Europe, la pollinisation représenterait près de 153 milliards d’euros soit 9,5\% de la valeur de production agricole mondiale. Les cultures dépendant des pollinisateurs, quant à elles, constituent près de 35\% en tonnes de la production mondiale de nourriture.

%Dans le même temps, d’autres insectes s’attaquent eux directement aux récoltes (directement aux champs, soit pendant le stockage) ou transmettent des maladies (virus, bactéries,...). Par exemple, Pimentel et al. [2005] évaluent les pertes liées aux insectes invasifs aux Etats-Unis (perte directe et coût de la lutte) à près de 20 milliards de dollars. Les ravageurs peuvent impacter toutes les parties de la plante : les feuilles, les tiges, les racines, les fleurs ou les fruits.... Les dégâts peuvent être tout autant quantitatifs que qualitatifs et se mesurent en unité absolue (poids, unité financière...) ou relative (en \% de production par exemple) \autocite{lebon:tel-01747618}. L'objet de cette étude porte sur l'un des ravageurs -pour ne pas dire le ravageur le plus dangereux- du maïs qui l'une des culture les plus consommées en Afrique. Ce ravageur porte le nom du \gls{cla} encore appelé \gls{spodo}.

La \gls{cla} est un lépidoptère nuisible originaire d'Amérique tropicale et subtropicale \cite{sparks1979}. Elle attaque plus de 80 espèces de cultures différentes, mais avec une préférence pour les graminées, et le maïs en particulier \autocite{day2017fall}. La présence du ravageur a été signalée en Afrique centrale et occidentale depuis 2016 \autocite{goergen2016first}, et plus tard dans la majeure partie de l'Afrique subsaharienne \autocite{day2017fall}. On ne sait pas comment cette invasion s'est produite, mais les preuves suggèrent que l'haplotype présent en Afrique est originaire de Floride et des Caraïbes \autocite{huesing2018integrated}. La prolificité de la \gls{cla} (les lots d'œufs contiennent souvent plusieurs centaines d'œufs ; \autocite{sparks1979}) associée à sa capacité à migrer sur de longues distances (plusieurs centaines de kilomètres ; \cite{rose1975migration}) sont deux des traits de l'espèce qui pourraient expliquer la vitesse à laquelle elle a envahi le continent \autocite{BAUDRON2019141}. La préférence du maïs ; la principale culture céréalière en Afrique \autocite{DEVI2018727} – et d'autres cultures dont se nourrit ce ravageur hautement polyphage – associée aux conditions agroécologiques propices à la \gls{cla} dans une grande partie de la région en fait une menace sérieuse pour la sécurité alimentaire en Afrique subsaharienne \autocite{day2017fall}.

Depuis l'invasion du continent par la \gls{cla}, la réaction immédiate des gouvernements a été d'investir dans les pesticides chimiques \autocite{HARRISON2019318} et leur utilisation reste la principale stratégie des agriculteurs pour lutter contre le ravageur, bien qu'avec des résultats mitigés \autocite{Kumela2019}. Plusieurs études on été publiées, mais il n'y a eu aucune étude systématique et quantitative à l'échelle nationale dans aucun des pays touchés en Afrique \autocite{DEGROOTE2020106804}. Les méthodes de lutte basées sur la gestion agronomique représentent une alternative intéressante, plus abordable pour les petits exploitants aux ressources limitées et à moindre risque pour la santé et l'environnement \autocite{thierfelder2018low}. Cependant, il existe peu de données empiriques pour orienter les recommandations pour un contrôle efficace de la \gls{cla} par la gestion agronomique en Afrique, car la plupart de ces connaissances sont basées sur des données provenant des Américains et des observations - parfois anecdotiques - faites dans la région \autocite{HARRISON2019318}.

L'impact de la \gls{cla} sur le rendement du maïs en Afrique a été signalé comme très important. \cite{day2017fall} ont estimé en moyenne l'impact de la \gls{cla} à 45\% (entre 22 et 67\%) du rendement au Ghana et à 40\% (entre 25 et 50\%) en Zambie, entraînant des millions de dollars US de pertes. De même, \cite{Kumela2019} ont estimé l'impact de la \gls{cla} à 32\% du rendement en Éthiopie et à 47\% du rendement au Kenya. Ces estimations, cependant, sont basées sur des enquêtes socio-économiques axées sur les perceptions des agriculteurs, mais pas sur des méthodes rigoureuses de dépistage sur le terrain telles que celle proposée par \cite{mcgrath2018monitoring}. Cependant, une étude directe  menée à Zimbabwe a estimé les perte causées par la \gls{cla} en 2018 à 11.57\%, mais seulement dans deux districts : Chipinge et Makoni \autocite{BAUDRON2019141}.

Face à cette pandémie, de nombreuses études ont été menées pour comprendre ces épidémies périodiques. Des modèles mathématiques basés sur les schémas météorologiques et les dates d'arrivée estimées des papillons à partir de la prévision des précipitations et du nombre de chenilles à partir des inspections sur le terrain ont également été formulés pour prédire d'éventuelles épidémies \autocite{Early391847, Bista2020}. Dans son étude, \cite{Early391847} a créé un ensemble de Modèles de Distribution des Espèces (SDM), comprenant huit techniques de modélisation : réseaux neuronaux artificiels (ANN), analyse des arbres de classification (CTA), analyse discriminante flexible (FDA), modèles additifs généralisés (GAM), modèles linéaires généralisés (GLM), splines de régression adaptative multivariée (MARS), forêt aléatoire (RF), et enveloppe de portée de surface (SRE).


\subsection{Problématique}
L'agriculture représente la part importante de source de vie de l'homme. Les ravageurs des cultures en générale et celui du maïs en particulier causent chaque année, des déficits alimentaires dûs aux pertes de récoltes. Des mesures de prévention et de contrôle ont été appliquées, mais la plus part est soit couteuse; donc difficile d'accès aux petits exploitant, soit présente un risque pour la santé et l'environnement. Il existe un riche ensemble de modèles mathématiques des effets de la \gls{cla} sur la production de maïs qui ont été rapportés par plusieurs auteurs \autocite{ANGUELOV2017437, Zhang2017, RAFIKOV2008557}. Des modèles tels que les modèles structurés par étapes avec et sans délai qui considèrent souvent les prédateurs matures (ennemis naturels) ont également été appliqués ; cependant, la plupart de ces modèles, bien qu'incorporant les types I, II, III et IV de Holling et les réponses fonctionnelles de Beddington-DeAngelis comme base de référence, n'ont pas ciblé l'interaction entre le maïs et les \gls{cla}.% [8, 16-24].%De plus, bon nombre de modèles mathématiques ont été utilisés pour comprendre le comportement du ravageur afin de mieux le contrôler
%Compte tenu de l'importance du maïs pour la majorité des pays producteurs de maïs en Afrique, le présent travail vise à

Cependant, dans une étude récente, \cite{Daudi2021} a utilisé une équation différentielle ordinaire pour explorer les implications de l'infestation par les \gls{cla} dans un champ de maïs planté avec un nombre initial de graines de maïs au temps $t = 0$.% et obtenant une récolte maximale à la fin de la saison.
Afin d'obtenir les résultats escomptés, il a proposé deux sous-modèles, chacun étant structuré par étapes dans les deux populations (maïs et \gls{cla}) pour déterminer la dynamique de la population en présence et en absence d'immigration de la \gls{cla} adulte et a estimé le rendement lorsque des mesures de contrôle telles que les pesticides et les cultures associées sont utilisées. L'un des défauts majeurs que nous avons fait ressortir de ce modèle, est le fait qu'il ne prenne pas en compte le fait qu'un maïs attaqué ne meurt pas nécessairement. Normalement, le maïs a la possibilité de compenser les pertes foliaires en fonction de la fertilité du sol \autocite{fao201812}. %que même en absence de ravageur, la population de maïs décroît de façon exponentielle. Très peu d'études sur les dégâts du ravageur ont montré un baisse de rendement de plus de 50\%. De plus, sur la base des études effectuées sur le terrain, les dégâts du ravageur deviennent de moins en moins important au fur et à mesure que la plante grandi. Ce qui nous permet de remettre en cause cette décroissance de la population de maïs. On comprend donc qu'il y a un certain seuil auquel va tendre vers la population de maïs dans le temps.
Nous proposerons alors une méthode qui prenne en compte ce facteur.
%L'un des défauts majeurs que nous avons fait ressortir de ce modèle, est le fait de pas prendre en compte le stade nymphale de l'insecte. En effet, ce stade est l'intermédiaire entre le stade larvaire et le stade adulte. A ce stade, le ravageur, devenu pupe, tombe du maïs, s'enfouit dans le sol de 6 à 8cm et y reste pendant 8 à 9 jours avant d'émerger pour devenir papillon (stade adulte \autocite{fao201812}). Il est donc évident qu'il ne se nourrit de maïs, ni ne se reproduit pendant ce temps. On peut donc voir ce stade comme une phase de transition entre le stade larvaire au cours duquel l'insecte se nourrit (détruit le maïs) et le stade adulte au cours duquel l'insecte se reproduit.

%\subsection{Organisation du travail}
%	Dans le cadre de notre étude, nous compte résoudre le problème en deux volées. La première consiste à faire une modélisation mathématique qui prenne

%Face à se telle situation, il est important de continuer à faire des investigations afin de parvenir à une approche consistante qui permettrait de lutter efficacement contre ce ravageur. Pour cela, notre étude consiste en la en la mise en place d'un modèle mathématique
%D'où l'objet de cette étude.


\subsection{Objectifs de l'étude}
\subsubsection{Objectif général}
Cette étude consiste à modéliser les relations entre dégâts de la chenille légionnaire, \gls{spodo} d'automne et le rendement du maïs au Bénin. Cela revient en quelque sorte à étudier la dynamique de la population du maïs et de celle du ravageur en fonction du temps.

\subsubsection{Objectifs spécifiques}
\begin{enumerate}[-]
	\item Modéliser à l'aide d'un système d'équations différentielles la dynamique de la population du maïs en présence de la \glsentrytext{cla} %de type,proie-prédateur
	%o* les deux populations sont structurées par étape, la variation de la population du maïs et celle de la \gls{cla} dans le temps,
	\item Estimer sur la base des données collectées sur le terrain, les paramètres du modèle,
	\item Résoudre  le système d'équations d'équation différentielle à l'aide d'une méthode de résolution numérique soigneusement choisie,
	\item Proposer une solution qui permettrait à tout producteur de prendre les mesures qu'il faut pour éviter le plus possible les dégâts de la \gls{cla}.
\end{enumerate}

%\subsection{Hypothèses de travail}
%	\begin{enumerate}[\it H1 :]
	%		\item La culture intercalaire utilisé (le niébé) atténue l'abondance de la \gls{cla} et réduit donc ses dégâts.
	%		\item L'abondance des arthropodes est liée à la présence de la culture intercalaire.
	%	\end{enumerate}





























%\section{Contexte et justification}
%La \gls{cla}, \gls{spodo} est un lépidoptère nuisible originaire d'Amérique tropicale et subtropicale \autocite{sparks1979}. Elle attaque plus de 80 espèces de cultures différentes, mais avec une préférence pour les graminées, et le maïs en particulier \autocite{day2017fall}. La présence du ravageur a été signalée en Afrique centrale et occidentale depuis 2016 \autocite{goergen2016first}, et plus tard dans la majeure partie de l'Afrique subsaharienne \autocite{day2017fall}. On ne sait pas comment cette invasion s'est produite, mais les preuves suggèrent que l'haplotype présent en Afrique est originaire de Floride et des Caraïbes \autocite{huesing2018integrated}. La prolificité de la \gls{cla} (les lots d'œufs contiennent souvent plusieurs centaines d'œufs ; \autocite{sparks1979}) associée à sa capacité à migrer sur de longues distances (plusieurs centaines de kilomètres ; \cite{rose1975migration}) sont deux des traits de l'espèce qui pourraient expliquer la vitesse à laquelle elle a envahi le continent \autocite{BAUDRON2019141}. La préférence du maïs ;la principale culture céréalière en Afrique \autocite{DEVI2018727} – et d'autres cultures dont se nourrit ce ravageur hautement polyphage – associée aux conditions agroécologiques propices à la \gls{cla} dans une grande partie de la région en fait une menace sérieuse (et très certainement pérenne) pour la sécurité alimentaire en Afrique subsaharienne \autocite{day2017fall}.
%
%Depuis l'invasion du continent par la \gls{cla}, la réaction immédiate des gouvernements a été d'investir dans les pesticides chimiques \autocite{HARRISON2019318} et leur utilisation reste la principale stratégie des agriculteurs pour lutter contre le ravageur, bien qu'avec des résultats mitigés \autocite{Kumela2019}. Plusieurs études on été publiées, mais il n'y a eu aucune étude systématique et quantitative à l'échelle nationale dans aucun des pays touchés en Afrique \autocite{DEGROOTE2020106804}. Les méthodes de lutte basées sur la gestion agronomique représentent une alternative intéressante, plus abordable pour les petits exploitants aux ressources limitées et à moindre risque pour la santé et l'environnement \autocite{thierfelder2018low}. Cependant, il existe peu de données empiriques pour orienter les recommandations pour un contrôle efficace de la \gls{cla} par la gestion agronomique en Afrique, car la plupart de ces connaissances sont basées sur des données provenant des Américains et des observations - parfois anecdotiques - faites dans la région \autocite{HARRISON2019318}.
%
%L'impact de la \gls{cla} sur le rendement du maïs en Afrique a été signalé comme très important. \cite{day2017fall} ont estimé en moyenne l'impact de la \gls{cla} à 45\% (entre 22 et 67\%) du rendement au Ghana et à 40\% (entre 25 et 50\%) en Zambie, entraînant des millions de dollars US de pertes. De même, \cite{Kumela2019} ont estimé l'impact de la \gls{cla} à 32\% du rendement en Éthiopie et à 47\% du rendement au Kenya. Ces estimations, cependant, sont basées sur des enquêtes socio-économiques axées sur les perceptions des agriculteurs, mais pas sur des méthodes rigoureuses de dépistage sur le terrain telles que celle proposée par \cite{mcgrath2018monitoring}. Cependant, une étude directe  menée à Zimbabwe a estimé les perte causées par la \gls{cla}s en 2018 à 11.57\%, mais seulement dans deux districts : Chipinge et Makoni \autocite{BAUDRON2019141}.
%
%Il est alors important de continuer à faire des investigations afin de parvenir à une approche consistante qui permettrait de lutter efficacement contre ce ravageur. D'où l'objet de cette étude.
%
%
%\section{Objectifs de l'étude}
%	\subsection{Objectif général}
%	Cette étude consiste à modéliser les relations entre dégâts de la chenille légionnaire, \gls{spodo} d'automne et le rendement du maïs au Bénin.
%
%	\subsection{Objectifs spécifiques}
%	\begin{enumerate}[-]
%		\item Estimer les dégâts causés par la \gls{cla} sur les plants de maïs ainsi que sur les épis,
%		\item Analyser l'impact du type de labour (labour minimum ou labour conventionnel) ainsi que celui du type de culture (monoculture de maïs ou inter culture de maïs et niébé) sur l'abondance de la \gls{cla},
%		\item Analyser l'impact du type de labour ainsi que celui du type de culture sur l'abondance des arthropodes.
%	\end{enumerate}
%
%	\section{Hypothèses de travail}
%	\begin{enumerate}[\it H1 :]
%		\item La culture intercalaire utilisé (le niébé) atténue l'abondance de la \gls{cla} et réduit donc ses dégâts.
%		\item L'abondance des arthropodes est liée à la présence de la culture intercalaire.
%	\end{enumerate}