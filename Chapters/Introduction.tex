\chapter{Introduction}
%\addstarredchapter{Introduction}
\minitoclt
%\setcounter{minitocdepth}{2}
%\minitoc

\section{Contexte et justification}
La \gls{cla}, \gls{spodo} est un lépidoptère nuisible originaire d'Amérique tropicale et subtropicale \autocite{sparks1979}. Elle attaque plus de 80 espèces de cultures différentes, mais avec une préférence pour les graminées, et le maïs en particulier \autocite{day2017fall}. La présence du ravageur a été signalée en Afrique centrale et occidentale depuis 2016 \autocite{goergen2016first}, et plus tard dans la majeure partie de l'Afrique subsaharienne \autocite{day2017fall}. On ne sait pas comment cette invasion s'est produite, mais les preuves suggèrent que l'haplotype présent en Afrique est originaire de Floride et des Caraïbes \autocite{huesing2018integrated}. La prolificité de la \gls{cla} (les lots d'œufs contiennent souvent plusieurs centaines d'œufs ; \autocite{sparks1979}) associée à sa capacité à migrer sur de longues distances (plusieurs centaines de kilomètres ; \cite{rose1975migration}) sont deux des traits de l'espèce qui pourraient expliquer la vitesse à laquelle elle a envahi le continent \autocite{BAUDRON2019141}. La préférence du maïs ;la principale culture céréalière en Afrique \autocite{DEVI2018727} – et d'autres cultures dont se nourrit ce ravageur hautement polyphage – associée aux conditions agroécologiques propices à la \gls{cla} dans une grande partie de la région en fait une menace sérieuse (et très certainement pérenne) pour la sécurité alimentaire en Afrique subsaharienne \autocite{day2017fall}.

Depuis l'invasion du continent par la \gls{cla}, la réaction immédiate des gouvernements a été d'investir dans les pesticides chimiques \autocite{HARRISON2019318} et leur utilisation reste la principale stratégie des agriculteurs pour lutter contre le ravageur, bien qu'avec des résultats mitigés \autocite{Kumela2019}. Plusieurs études on été publiées, mais il n'y a eu aucune étude systématique et quantitative à l'échelle nationale dans aucun des pays touchés en Afrique \autocite{DEGROOTE2020106804}. Les méthodes de lutte basées sur la gestion agronomique représentent une alternative intéressante, plus abordable pour les petits exploitants aux ressources limitées et à moindre risque pour la santé et l'environnement \autocite{thierfelder2018low}. Cependant, il existe peu de données empiriques pour orienter les recommandations pour un contrôle efficace de la \gls{cla} par la gestion agronomique en Afrique, car la plupart de ces connaissances sont basées sur des données provenant des Américains et des observations - parfois anecdotiques - faites dans la région \autocite{HARRISON2019318}.

L'impact de la \gls{cla} sur le rendement du maïs en Afrique a été signalé comme très important. \cite{day2017fall} ont estimé en moyenne l'impact de la \gls{cla} à 45\% (entre 22 et 67\%) du rendement au Ghana et à 40\% (entre 25 et 50\%) en Zambie, entraînant des millions de dollars US de pertes. De même, \cite{Kumela2019} ont estimé l'impact de la \gls{cla} à 32\% du rendement en Éthiopie et à 47\% du rendement au Kenya. Ces estimations, cependant, sont basées sur des enquêtes socio-économiques axées sur les perceptions des agriculteurs, mais pas sur des méthodes rigoureuses de dépistage sur le terrain telles que celle proposée par \cite{mcgrath2018monitoring}. Cependant, une étude directe  menée à Zimbabwe a estimé les perte causées par la \gls{cla}s en 2018 à 11.57\%, mais seulement dans deux districts : Chipinge et Makoni \autocite{BAUDRON2019141}.

Il est alors important de continuer à faire des investigations afin de parvenir à une approche consistante qui permettrait de lutter efficacement contre ce ravageur. D'où l'objet de cette étude.


\section{Objectifs de l'étude}
	\subsection{Objectif général}
	Cette étude consiste à modéliser les relations entre dégâts de la chenille légionnaire, \gls{spodo} d'automne et le rendement du maïs au Bénin.

	\subsection{Objectifs spécifiques}
	\begin{enumerate}[-]
		\item Estimer les dégâts causés par la \gls{cla} sur les plants de maïs ainsi que sur les épis,
		\item Analyser l'impact du type de labour (labour minimum ou labour conventionnel) ainsi que celui du type de culture (monoculture de maïs ou inter culture de maïs et niébé) sur l'abondance de la \gls{cla},
		\item Analyser l'impact du type de labour ainsi que celui du type de culture sur l'abondance des arthropodes.
	\end{enumerate}

	\section{Hypothèses de travail}
	\begin{enumerate}[\it H1 :]
		\item La culture intercalaire utilisé (le niébé) atténue l'abondance de la \gls{cla} et réduit donc ses dégâts.
		\item L'abondance des arthropodes est liée à la présence de la culture intercalaire.
	\end{enumerate}